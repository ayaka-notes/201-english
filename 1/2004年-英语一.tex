\bta{2004}


\section{Use of English}

\noindent
\textbf{Directions:}\\
Read the following text. Choose the best word (s) for each numbered blank
and mark A, B, C or D on ANSWER SHEET 1. (10 points)


\TiGanSpace


Many theories concerning the causes of juvenile delinquency (crimes
committed by young people) focus either on the individual or on society
as the major contributing influence. Theories \cloze on the
individual suggest that children engage in criminal behavior
\cloze they were not sufficiently penalized for previous misdeeds
or that they have learned criminal behavior through \cloze with
others. Theories focusing on the role of society suggest that children
commit crimes in \cloze to their failure to rise above their
socioeconomic status, \cloze as a rejection of middle-class
values.

Most theories of juvenile delinquency have focused on children from
disadvantaged families, \cloze the fact that children from
wealthy homes also commit crimes. The latter may commit crimes
\cloze lack of adequate parental control. All theories, however,
are tentative and are \cloze to criticism.

Changes in the social structure may indirectly \cloze juvenile
crime rates. For example, changes in the economy that \cloze to
fewer job opportunities for youth and rising unemployment \cloze
make gainful employment increasingly difficult to obtain. The resulting
discontent may in \cloze lead more youths into criminal
behavior.

Families have also \cloze changes these years. More families
consist of one-parent households or two working parents; \cloze , children are likely to have less supervision at home \cloze
was common in the traditional family \cloze. This lack of
parental supervision is thought to be an influence on juvenile crime
rates. Other \cloze causes of offensive acts include
frustration or failure in school, the increased \cloze
of drugs and alcohol, and the growing \cloze of child abuse and
child neglect. All these conditions tend to increase the probability of
a child committing a criminal act, \cloze a direct causal
relationship has not yet been established.


\newpage
\begin{enumerate}
	%\renewcommand{\labelenumi}{\arabic{enumi}.}
	% A(\Alph) a(\alph) I(\Roman) i(\roman) 1(\arabic)
	%设定全局标号series=example	%引用全局变量resume=example
	%[topsep=-0.3em,parsep=-0.3em,itemsep=-0.3em,partopsep=-0.3em]
	%可使用leftmargin调整列表环境左边的空白长度 [leftmargin=0em]
	\item


\fourchoices
{acting}
{relying}
{centering}
{commenting}




\item


\fourchoices
{before}
{unless}
{until}
{because}




\item

\fourchoices
{interaction}
{assimilation}
{cooperation}
{consultation}


\item


\fourchoices
{return}
{reply}
{reference}
{response}




\item
\fourchoices
{or}
{but rather}
{but}
{or else}



\item

\fourchoices
{considering}
{ignoring}
{highlighting}
{discarding}



\item


\fourchoices
{on}
{in}
{for}
{with}




\item


\fourchoices
{immune}
{resistant}
{sensitive}
{subject}




\item


\fourchoices
{affect}
{reduce}
{chock}
{reflect}




\item


\fourchoices
{point}
{lead}
{come}
{amount}




\item


\fourchoices
{in general}
{on average}
{by contrast}
{at length}





\item


\fourchoices
{case}
{short}
{turn}
{essence}




\item

\fourchoices
{survived}
{noticed}
{undertaken}
{experienced}



\item

\fourchoices
{contrarily}
{consequently}
{similarly}
{simultaneously}


\item


\fourchoices
{than}
{that}
{which}
{as}




\item


\fourchoices
{system}
{structure}
{concept}
{heritage}




\item

\fourchoices
{assessable}
{identifiable}
{negligible}
{incredible}


\item

\fourchoices
{expense}
{restriction}
{allocation}
{availability}



\item

\fourchoices
{incidence}
{awareness}
{exposure}
{popularity}



\item


\fourchoices
{provided}
{since}
{although}
{supposing}



\end{enumerate}


\hfil

\section{Reading Comprehension}



\noindent
\textbf{Part A}\\
\textbf{Directions:}\\
Read the following four texts. Answer the questions below each
	text by choosing A, B, C or
	D. Mark your answers
	on ANSWER SHEET 1. (40 points)

\newpage
\subsection{Text 1}


Hunting for a job late last year, lawyer Gant Redmon stumbled across
CareerBuilder, a job database on the Internet. He searched it with no
success but was attracted by the site's ``personal search agent''. It's
an interactive feature that lets visitors key in job criteria such as
location, title, and salary, then E-mails them when a matching position
is posted in the database. Redmon chose the keywords \emph{legal},
	\emph{intellectual property} and \emph{Washington, D.C.} Three weeks later, he
got his first notification of an opening. ``I struck gold,'' says
Redmon, who E-mailed his resume to the employer and won a position as
in-house counsel for a company.

With thousands of career-related sites on the Internet, finding
promising openings can he time-consuming and inefficient. Search agents
reduce the need for repeated visits to the databases. But although a
search agent worked for Redmon, career experts see drawbacks. Narrowing
your criteria, for example, may work against you: ``Every time you
answer a question you eliminate a possibility,'' says one expert.

For any job search, you should start with a narrow concept---what you
think you want to do---then broaden it. ``None of these programs do
that,'' says another expert. ``There's no career counseling implicit in
all of this.'' Instead, the best strategy is to use the agent as a kind
of \uline{tip service} to keep abreast of jobs in a particular
database; when you get E-mail, consider it a reminder to check the
database again. ``I would not rely on agents for finding everything that
is added to a database that might interest me,'' says the author of a
job-searching guide.

Some sites design their agents to tempt job hunters to return. When
CareerSite's agent sends out messages to those who have signed up for
its service, for example, it includes only three potential jobs---those
it considers the best matches. There may be more matches in the
database; job hunters will have to visit the site again to find
them---and they do. ``On the day after we send our messages, we see a
sharp increase in our traffic,'' says Seth Peets, vice president of
marketing for CareerSite.

Even those who aren't hunting for jobs may find search agents
worthwhile. Some use them to keep a close watch on the demand for their
line of work or gather information on compensation to arm themselves
when negotiating for a raise. Although happily employed, Redmon
maintains his agent at CareerBuilder. ``You always keep your eyes
open,'' he says. Working with a personal search agent means having
another set of eyes looking out for you.


\begin{enumerate}[resume]
	%\renewcommand{\labelenumi}{\arabic{enumi}.}
	% A(\Alph) a(\alph) I(\Roman) i(\roman) 1(\arabic)
	%设定全局标号series=example	%引用全局变量resume=example
	%[topsep=-0.3em,parsep=-0.3em,itemsep=-0.3em,partopsep=-0.3em]
	%可使用leftmargin调整列表环境左边的空白长度 [leftmargin=0em]
	\item
How did Redmon find his job?


\fourchoices
{By searching openings in a job database.}
{By posting a matching position in a database.}
{By using a special service of a database.}
{By E-mailing his resume to a database.}


\item
Which of the following can be a disadvantage of search agents?


\fourchoices
{Lack of counseling.}
{Limited number of visits.}
{Lower efficiency.}
{Fewer successful matches.}


\item
The expression ``tip service'' (Line 4, Paragraph 3) most probably
means \lineread.


\fourchoices
{advisory}
{compensation}
{interaction}
{reminder}


\item
Why does CareerSite's agent offer each job hunter only three job
options?


\fourchoices
{To focus on better job matches.}
{To attract more returning visits.}
{To reserve space for more messages.}
{To increase the rate of success.}


\item
Which of the following is true according to the text?

\fourchoices
{Personal search agents are indispensable to job-hunters.}
{Some sites keep E-mailing job seekers to trace their demands.}
{Personal search agents are also helpful to those already}
{Some agents stop sending information to people once they are}


\end{enumerate}

	
\newpage
\subsection{Text 2}
	


Over the past century, all kinds of unfairness and discrimination have
been condemned or made illegal. But one insidious form continues to
thrive: alphabetism. This, for those as yet unaware of such a
disadvantage, refers to discrimination against those whose surnames
begin with a letter in the lower half of the alphabet.

It has long been known that a taxi firm called AAAA cars has a big
advantage over Zodiac cars when customers thumb through their phone
directories. Less well known is the advantage that Adam Abbott has in
life over Zoë Zysman. English names are fairly evenly spread between the
halves of the alphabet. Yet a suspiciously large number of top people
have surnames beginning with letters between A and K.

Thus the American president and vice-president have surnames starting
with B and C respectively; and 26 of George Bush's predecessors
(including his father) had surnames in the first half of the alphabet
against just 16 in the second half. Even more striking, six of the seven
heads of government of the G 7 rich countries are alphabetically
advantaged (Berlusconi, Blair, Bush, Chirac, Chrétien and Koizumi). The
world's three top central bankers (Greenspan, Duisenberg and Hayami) are
all close to the top of the alphabet, even if one of them really uses
Japanese characters. As are the world's five richest men (Gates,
Buffett, Allen, Ellison and Albrecht).

Can this merely be coincidence? One theory, dreamt up in all the spare
time enjoyed by the alphabetically disadvantaged, is that the rot sets
in early. At the start of the first year in infant school, teachers seat
pupils alphabetically from the front, to make it easier to remember
their names. So short-sighted Zysman junior gets stuck in the back row,
and is rarely asked the improving questions posed by those insensitive
teachers. At the time the alphabetically disadvantaged may think they
have had a lucky escape. Yet the result may be worse qualifications,
because they get less individual attention, as well as less confidence
in speaking publicly.

The humiliation continues. At university graduation ceremonies, the ABCs
proudly get their awards first; by the time they reach the Zysmans \uline{most
people are literally having a ZZZ}. Shortlists for job interviews,
election ballot papers, lists of conference speakers and attendees: all
tend to be drawn up alphabetically, and their recipients lose interest
as they plough through them.

\newpage
\begin{enumerate}[resume]
	%\renewcommand{\labelenumi}{\arabic{enumi}.}
	% A(\Alph) a(\alph) I(\Roman) i(\roman) 1(\arabic)
	%设定全局标号series=example	%引用全局变量resume=example
	%[topsep=-0.3em,parsep=-0.3em,itemsep=-0.3em,partopsep=-0.3em]
	%可使用leftmargin调整列表环境左边的空白长度 [leftmargin=0em]
	\item
What does the author intend to illustrate with AAAA cars and Zodiac
cars?


\fourchoices
{A kind of overlooked inequality.}
{A type of conspicuous bias.}
{A type of personal prejudice.}
{A kind of brand discrimination.}

 
\item
 What can we infer from the first three paragraphs?

\fourchoices
{In both East and West, names are essential to success.}
{The alphabet is to blame for the failure of Zoë Zysman.}
{Customers often pay a lot of attention to companies' names.}
{Some form of discrimination is too subtle to recognize.}



\item
The 4th paragraph suggests that \lineread.

\fourchoices
{questions are often put to the more intelligent students}
{alphabetically disadvantaged students often escape from class}
{teachers should pay attention to all of their students}
{students should be seated according to their eyesight}


\item
What does the author mean by ``most people are literally having a
ZZZ'' (Lines 2$ \sim $3, Paragraph 5)?


\fourchoices
{They are getting impatient.}
{They are noisily dozing off.}
{They are feeling humiliated.}
{They are busy with word puzzles.}


\item
Which of the following is true according to the text?

\fourchoices
{People with surnames beginning with N to Z are often ill-treated.}
{VIPs in the Western world gain a great deal from alphabetism.}
{The campaign to eliminate alphabetism still has a long way to go.}
{Putting things alphabetically may lead to unintentional bias.}

\end{enumerate}


\newpage
\subsection{Text 3}



When it comes to the slowing economy, \uline{Ellen Spero isn't biting her nails
just yet}. But the 47-year-old manicurist isn't cutting, filing or
polishing as many nails as she'd like to, either. Most of her clients
spend \$12 to \$50 weekly, but last month two longtime customers
suddenly stopped showing up. Spero blames the softening economy. ``I'm a
good economic indicator,'' she says. ``I provide a service that people
can do without when they're concerned about saving some dollars.'' So
Spero is downscaling, shopping at middle-brow Dillard's department store
near her suburban Cleveland home, instead of Neiman Marcus. ``I don't
know if other clients are going to abandon me, too,'' she says.

Even before Alan Greenspan's admission that America's red-hot economy is
cooling, lots of working folks had already seen signs of the slowdown
themselves. From car dealerships to Gap outlets, sales have been lagging
for months as shoppers temper their spending. For retailers, who last
year took in 24 percent of their revenue between Thanksgiving and
Christmas, the cautious approach is coming at a crucial time. Already,
experts say, holiday sales are off 7 percent from last year's pace. But
don't sound any alarms just yet. Consumers seem only mildly concerned,
not panicked, and many say they remain optimistic about the economy's
long-term prospects even as they do some modest belt-tightening.

Consumers say they're not in despair because, despite the dreadful
headlines, their own fortunes still feel pretty good. Home prices are
holding steady in most regions. In Manhattan, ``there's a new gold rush
happening in \uline{the \$4 million to \$10 million range}, predominantly fed by
Wall Street bonuses,'' says broker Barbara Corcoran. In San Francisco,
prices are still rising even as frenzied overbidding quiets. ``Instead
of 20 to 30 offers, now maybe you only get two or three," says John
Tealdi, a Bay Area real-estate broker. And most folks still feel pretty
comfortable about their ability to find and keep a job.

Many folks see silver linings to this slowdown. Potential home buyers
would cheer for lower interest rates. Employers wouldn't mind a little
fewer bubbles in the job market. Many consumers seem to have been
influenced by stock-market swings, which investors now view as a
necessary ingredient to a sustained boom. Diners might see an upside,
too. Getting a table at Manhattan's hot new Alain Ducasse restaurant
used to be impossible. Not anymore. For that, Greenspan \& Co. may still
be worth toasting.

\begin{enumerate}[resume]
	%\renewcommand{\labelenumi}{\arabic{enumi}.}
	% A(\Alph) a(\alph) I(\Roman) i(\roman) 1(\arabic)
	%设定全局标号series=example	%引用全局变量resume=example
	%[topsep=-0.3em,parsep=-0.3em,itemsep=-0.3em,partopsep=-0.3em]
	%可使用leftmargin调整列表环境左边的空白长度 [leftmargin=0em]
	\item
By ``Ellen Spero isn't biting her nails just yet'' (Line 1,
Paragraph 1), the author means \lineread.


\fourchoices
{Spero can hardly maintain her business}
{Spero is too much engaged in her work}
{Spero has grown out of her bad habit}
{Spero is not in a desperate situation}


\item
How do the public feel about the current economic situation?


\fourchoices
{Optimistic.}
{Confused.}
{Carefree.}
{Panicked.}


\item
 When mentioning ``the \$4 million to \$10 million range'' (Lines 3,
Paragraph 3), the author is talking about \lineread.

\fourchoices
{gold market}
{real estate}
{stock exchange}
{venture investment}


\item
Why can many people see ``silver linings'' to the economic slowdown?


\fourchoices
{They would benefit in certain ways.}
{The stock market shows signs of recovery.}
{Such a slowdown usually precedes a boom.}
{The purchasing power would be enhanced.}


\item
To which of the following is the author likely to agree?


\fourchoices
{A new boom, on the horizon.}
{Tighten the belt, the single remedy.}
{Caution all right, panic not.}
{The more ventures, the more chances.}



\end{enumerate}


\newpage
\subsection{Text 4}


Americans today don't place a very high value on intellect. Our heroes
are athletes, entertainers, and entrepreneurs, not scholars. Even our
schools are where we send our children to get a practical
education---not to pursue knowledge for the sake of knowledge. Symptoms
of pervasive anti-intellectualism in our schools aren't difficult to
find.

``Schools have always been in a society where practical is more
important than intellectual,'' says education writer Diane Ravitch.
``Schools could be a counterbalance.'' Ravitch's latest book. \emph{Left
	Back: A Century of Failed School Reforms}, traces the roots of
anti-intellectualism in our schools, concluding they are anything but a
counterbalance to the American distaste for intellectual pursuits.

But they could and should be. Encouraging kids to reject the life of the
mind leaves them vulnerable to exploitation and control. Without the
ability to think critically, to defend their ideas and understand the
ideas of others, they cannot fully participate in our democracy.
Continuing along this path, says writer Earl Shorris, ``We will become a
second-rate country. We will have a less civil society.''

``Intellect is resented as a form of power or privilege,'' writes
historian and professor Richard Hofstadter in \emph{Anti-intellectualism
	in American Life,} a Pulitzer-Prize winning book on the roots of
anti-intellectualism in US politics, religion, and education. From the
beginning of our history, says Hofstadter, our democratic and populist
urges have driven us to reject anything that smells of elitism.
Practicality, common sense, and native intelligence have been considered
more noble qualities than anything you could learn from a book.

Ralph Waldo Emerson and other Transcendentalist philosophers thought
schooling and rigorous book learning put unnatural restraints on
children: ``We are shut up in schools and college recitation rooms for
10 or 15 years and come out at last with a bellyful of words and do not
know a thing.''Mark Twain's \emph{Huckleberry Finn} exemplified American
anti-intellectualism. Its hero avoids being civilized---going to school
and learning to read---so he can preserve his innate goodness.

Intellect, according to Hofstadter, is different from native
intelligence, a quality we reluctantly admire. Intellect is the
critical, creative, and contemplative side of the mind. Intelligence
seeks to grasp, manipulate, re-order, and adjust, while intellect
examines, ponders, wonders, theorizes, criticizes, and imagines.

School remains a place where intellect is mistrusted. Hofstadter says
our country's educational system is in the grips of people who
``joyfully and militantly proclaim their hostility to intellect and
their eagerness to identify with children who show the least
intellectual promise.''


\begin{enumerate}[resume]
	%\renewcommand{\labelenumi}{\arabic{enumi}.}
	% A(\Alph) a(\alph) I(\Roman) i(\roman) 1(\arabic)
	%设定全局标号series=example	%引用全局变量resume=example
	%[topsep=-0.3em,parsep=-0.3em,itemsep=-0.3em,partopsep=-0.3em]
	%可使用leftmargin调整列表环境左边的空白长度 [leftmargin=0em]
	\item
 What do American parents expect their children to acquire in school?


\fourchoices
{The habit of thinking independently.}
{Profound knowledge of the world.}
{Practical abilities for future career.}
{The confidence in intellectual pursuits.}


\item
We can learn from the text that Americans have a history
of \lineread.


\fourchoices
{undervaluing intellect}
{favoring intellectualism}
{supporting school reform}
{suppressing native intelligence}


\item
The views of Raviteh and Emerson on schooling are \lineread.


\fourchoices
{identical}
{similar}
{complementary}
{opposite}


\item
Emerson, according to the text, is probably \lineread.


\fourchoices
{a pioneer of education reform}
{an opponent of intellectualism}
{a scholar in favor of intellect}
{an advocate of regular schooling}



\item
What does the author think of intellect?


\fourchoices
{It is second to intelligence.}
{It evolves from common sense.}
{It is to be pursued.}
{It underlies power}


\end{enumerate}


\newpage

\noindent
\textbf{Part B}\\
\textbf{Directions:}\\
Read the following text carefully and then translate the
	underlined segments into Chinese. Your translation should be written
	clearly on ANSWER SHEET 2. (10 points)


\TiGanSpace


The relation of language and mind has interested philosophers for many
centuries. \transnum \uline{The Greeks assumed that the structure of
	language had some connection with the process of thought, which took
	root in Europe long before people realized how diverse languages could
	be.}

Only recently did linguists begin the serious study of languages that
were very different from their own. Two anthropologist-linguists, Franz
Boas and Edward Sapir, were pioneers in describing many native languages
of North and South America during the first half of the twentieth
century. \transnum \uline{We are obliged to them because some of these
	languages have since vanished, as the peoples who spoke them died out or
	became assimilated and lost their native languages.} Other linguists in
the earlier part of this century, however, who were less eager to deal
with bizarre data from ``exotic'' language, were not always so grateful.
\transnum \uline{The newly described languages were often so strikingly
	different from the well studied languages of Europe and Southeast Asia
	that some scholars even accused Boas and Sapir of fabricating their
	data.} Native American languages are indeed different, so much so in
fact that Navajo could be used by the US military as a code during World
War II to send secret messages.

Sapir's pupil, Benjamin Lee Whorf, continued the study of American
Indian languages. \transnum \uline{Being interested in the relationship
	of language and thought, Whorf developed the idea that the structure of
	language determines the structure of habitual thought in a society}. He
reasoned that because it is easier to formulate certain concepts and not
others in a given language, the speakers of that language think along
one track and not along another. \transnum \uline{Whorf came to believe
	in a sort of linguistic determinism which, in its strongest form, states
	that language imprisons the mind, and that the grammatical patterns in a
	language can produce far-reaching consequences for the culture of a
	society}. Later, this idea became to be known as the Sapir-Whorf
hypothesis, but this term is somewhat inappropriate. Although both Sapir
and Whorf emphasized the diversity of languages, Sapir himself never
explicitly supported the notion of linguistic determinism.



\section{Writing}


\textbf{46. Directions:}

Study the following drawing carefully and write an essay in
	which you should
\begin{listwrite}
\item 
 describe the drawing,



\item
 interpret its meaning, and



\item
support your view with examples.
\end{listwrite}

You should write about 200 words neatly on ANSWER SHEET 2. (20 points)


\begin{figure}[h!]
	\centering
	\includesvg[width=0.5\linewidth]{picture/svg/picture-03}
	\caption*{终点又是新起点}
\end{figure}


\checkpagenumber
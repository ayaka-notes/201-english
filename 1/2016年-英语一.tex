\bta{2016}



\section{Use of English}

\noindent
\textbf{ Directions:}\\
Read the following text. Choose the best word (s) for each
	numbered blank and mark A, B, C or D on the \textbf{ANSWER SHEET}. (10 points)


\TiGanSpace

In Cambodia, the choice of a spouse is a complex one for the young
male. It may involve not only his parents and his
friends, \cloze those of the young women, but also a
matchmaker. A young man can \cloze a likely spouse on his own
and them ask his parents to \cloze the marriage negotiations, or
the young man's parents may make the choice of a spouse, giving the
child little to say in the selection. \cloze , a girl may veto
the spouse her parents have chosen. \cloze a spouse has been
selected, each family investigates the other to make sure its child is
marrying \cloze a good family.

The traditional wedding is a long and colorful affair. Formerly it
lasted three days, \cloze by the 1980s it more commonly lasted a
day and a half. Buddhist priests offer a short sermon and \cloze
prayers of blessing. Parts of the ceremony involve ritual hair
cutting, \cloze cotton threads soaked in holy water around the
bride's and groom's wrists , and \cloze a candle around a circle
of happily married and respected couples to bless the \cloze
. Newlyweds traditionally move in with the wife's parents and may
\cloze with them up to a year, \cloze they can build a
flew house nearby.

Divorce is legal and easy to \cloze , but not common. Divorced
persons are \cloze with some disapproval. Each spouse retains
\cloze property he or she \cloze into the marriage, and
jointly acquired property is \cloze equally. Divorced persons
may remarry, but a gender prejudice \cloze up. The divorced
male doesn't have a waiting period before he can remarry \cloze
the woman must wait the months.


\newpage

\begin{enumerate}
	%\renewcommand{\labelenumi}{\arabic{enumi}.}
	% A(\Alph) a(\alph) I(\Roman) i(\roman) 1(\arabic)
	%设定全局标号series=example	%引用全局变量resume=example
	%[topsep=-0.3em,parsep=-0.3em,itemsep=-0.3em,partopsep=-0.3em]
	%可使用leftmargin调整列表环境左边的空白长度 [leftmargin=0em]
	\item

\fourchoices
{by way of}
{as well as}
{on behalf of}
{with regard to}


\item

\fourchoices
{adapt to}
{provide for}
{compete with}
{decide on}



\item


\fourchoices
{close}
{renew}
{arrange}
{postpone}




\item


\fourchoices
{In theory}
{Above all}
{In time}
{For example}





\item


\fourchoices
{Although}
{Lest}
{After}
{Unless}




\item


\fourchoices
{into}
{within}
{from}
{through}




\item

\fourchoices
{since}
{or}
{but}
{so}



\item


\fourchoices
{test}
{copy}
{recite}
{create}




\item


\fourchoices
{folding}
{piling}
{wrapping}
{tying}




\item


\fourchoices
{lighting}
{passing}
{hiding}
{serving}




\item

\fourchoices
{meeting}
{association}
{collection}
{union}




\item


\fourchoices
{grow}
{part}
{deal}
{live}




\item


\fourchoices
{whereas}
{until}
{for}
{if}




\item


\fourchoices
{obtain}
{follow}
{challenge}
{avoid}




\item

\fourchoices
{isolated}
{persuaded}
{viewed}
{exposed}



\item

\fourchoices
{wherever}
{however}
{whenever}
{whatever}


\item


\fourchoices
{changed}
{brought}
{shaped}
{pushed}




\item

\fourchoices
{divided}
{invested}
{donated}
{withdrawn}


\item


\fourchoices
{clears}
{warms}
{shows}
{breaks}




\item


\fourchoices
{while}
{so what}
{once}
{in that}


\end{enumerate}

\vfil

\section{Reading Comprehension}


\noindent
\textbf{Part A}\\
\textbf{ Directions:}\\
 Read the following four texts. Answer the questions below
	each text by choosing A, B, C or
	D. Mark your answers on the \textbf{ANSWER SHEET}. (40 points)

\newpage
\subsection{Text 1}


France, which prides itself as the global innovator of fashion,
has decided its fashion industry has lost an absolute right to define
physical beauty for woman. Its lawmakers gave preliminary approval last
week to a law that would make it a crime to employ ultra-thin models on
runways. The parliament also agreed to ban websites that ``incite
excessive thinness'' by promoting extreme dieting.

Such measures have a couple of uplifting motives. They suggest
beauty should not be defined by looks that end up \uline{impinging on} health.
That's a start. And the ban on ultra-thin models seems to go beyond
protecting models from starving themselves to death---as some have
done. It tells the fashion industry that it must take responsibility for
the signal it sends women, especially teenage girls, about the social
tape--measure they must use to determine their individual worth.

The bans, if fully enforced , would suggest to woman (and many men) that they should not let others be arbiters of their beauty. And
perhaps faintly, they hint that people should look to intangible
qualities like character and intellect rather than dieting their way to
size zero or wasp-waist physiques.

The French measures, however, rely too much on severe punishment to
change a culture that still regards beauty as skin-deep---and
bone-showing. Under the law, using a fashion model that does not meet
a government-defined index of body mass could result in a \$85, 000 fine
and six months in prison.

The fashion industry knows it has an inherent problem in focusing on
material adornment and idealized body types. In Denmark, the United
States, and a few other countries, it is trying to set voluntary
standard for models and fashion images that rely more on peer pressure
for enforcement.

In contrast to France's actions, Denmark's fashion industry agreed last
month on rules and sanctions regarding the age, health, and other
characteristics of models. The newly revised Danish Fashion Ethical
Charter clearly states: ``We are aware of and take responsibility for
the impact the fashion industry has on body ideals, especially on young
people.'' The charter's main tool of enforcement is to deny access for
designers and modeling agencies to Copenhagen Fashion Week (CFW), which is
run by the Danish Fashion Institute. But in general it relies on a
name-and-shame method of compliance.

Relying on ethical persuasion rather than law to address the misuse of
body ideals may be the best step. Even better would be to help elevate
notions of beauty beyond the material standards of a particular
industry.


\begin{enumerate}[resume]
	%\renewcommand{\labelenumi}{\arabic{enumi}.}
	% A(\Alph) a(\alph) I(\Roman) i(\roman) 1(\arabic)
	%设定全局标号series=example	%引用全局变量resume=example
	%[topsep=-0.3em,parsep=-0.3em,itemsep=-0.3em,partopsep=-0.3em]
	%可使用leftmargin调整列表环境左边的空白长度 [leftmargin=0em]
	\item
According to the first paragraph, what would happen in
France?


\fourchoices
{New runways would be constructed}
{Physical beauty would be redefined}
{Websites about dieting would thrive}
{The fashion industry would decline}



\item
The phrase ``impinging on'' (Line 2 Para. 2) is closest in
meaning to \lineread.


\fourchoices
{heightening the value of}
{indicating the state of}
{losing faith in}
{doing harm to}


\item
Which of the following is true of the fashion
industry?


\fourchoices
{New standards are being set in Denmark}
{The French measures have already failed}
{Models are no longer under peer pressure}
{Its inherent problems are getting worse}



\item
A designer is most likely to be rejected by CFW for \lineread.


\fourchoices
{pursuing perfect physical conditions}
{caring too much about models' character}
{showing little concern for health factors}
{setting a high age threshold for models}


\item
Which of the following may be the best title of the
text?


\fourchoices
{A Challenge to the Fashion Industry's Body Ideals}
{A Dilemma for the Starving Models in France}
{Just Another Round of Struggle for Beauty}
{The Great Threats to the Fashion Industry}


\end{enumerate}


\newpage
\subsection{Text 2}


For the first time in the history more people live in towns than in the
country. In Britain this has had a curious result. While polls show
Britons rate ``the countryside'' alongside the royal family.
Shakespeare and the National Health Service (NHS) as what makes them
proudest of their country, this has limited political support.

A century ago Octavia Hill launched the National Trust not to rescue
stylish houses but to save ``the beauty of natural places for everyone
forever''. It was specifically to provide city dwellers with spaces for
leisure where they could experience ``a refreshing air''. Hill's
pressure later led to the creation of national parks and green belts.
They don't make countryside any more, and every year concrete consumes
more of it. It needs constant guardianship.

At the next election none of the big parties seem likely to endorse
this sentiment. The Conservatives' planning reform explicitly gives
rural development priority over conservation, even authorizing
``off--plan'' building where local people might object. The concept of
sustainable development has been defined as profitable. Labour likewise
wants to discontinue local planning where councils oppose development.
The Liberal Democrats are silent. Only Ukip, sensing its chance, has
sided with those pleading for a more considered approach to using green
land. Its Campaign to Protect Rural England struck terror into many
local Conservative parties.

The sensible place to build new houses, factories and offices is
where people are, in cities and towns where infrastructure is in
place. The London agents Stirling Ackroyd recently identified enough
sites for half a million houses in the London area alone, with no
intrusion on green belts. What is true of London is even truer of the
provinces.

The idea that ``housing crisis'' equals ``concreted meadows'' is pure
lobby talk. The issue is not the need for more houses but, as always,
where to put them. Under lobby pressure, George Osborne favours rural
new-build against urban renovation and renewal. He favours out-of-town
shopping sites against high streets. This is not a free market but a
biased one. Rural towns and villages have grown and will always grow.
They do so best where building sticks to their edges and respects their
character. We do not ruin urban conservation areas. Why ruin rural
ones?

Development should be planned, not let rip, After the
Netherlands, Britain is Europe's most crowded country. Half a century
of town and country planning has enabled it to retain an enviable rural
coherence, while still permitting low-density urban living. There is
no doubt of the alternative---the corrupted landscapes of southern
Portugal, Spain or Ireland. Avoiding this rather than promoting it
should unite the left and right of the political spectrum.


\begin{enumerate}[resume]
	%\renewcommand{\labelenumi}{\arabic{enumi}.}
	% A(\Alph) a(\alph) I(\Roman) i(\roman) 1(\arabic)
	%设定全局标号series=example	%引用全局变量resume=example
	%[topsep=-0.3em,parsep=-0.3em,itemsep=-0.3em,partopsep=-0.3em]
	%可使用leftmargin调整列表环境左边的空白长度 [leftmargin=0em]
	\item
 Britain's public sentiment about the countryside \lineread.


\fourchoices
{didn't start till the Shakespearean age}
{has brought much benefit to the NHS}
{is fully backed by the royal family}
{is not well reflected in politics}




\item
According to Paragraph 2, the achievements of the National
Trust are now being \lineread.


\fourchoices
{gradually destroyed}
{effectively reinforced}
{largely overshadowed}
{properly protected}



\item
Which of the following can be inferred from Paragraph 3?


\fourchoices
{Labour is under attack for opposing development}
{The Conservatives may abandon ``off-plan'' building}
{The Liberal Democrats are losing political influence}
{Ukip may gain from its support for rural conservation}


\item
The author holds that George Osbornes's preference \lineread.


\fourchoices
{highlights his firm stand against lobby pressure}
{shows his disregard for the character of rural areas}
{stresses the necessity of easing the housing crisis}
{reveals a strong prejudice against urban areas}



\item
 In the last paragraph the author shows his appreciation of \lineread.


\fourchoices
{the size of population in Britain}
{the political life in today's Britain}
{the enviable urban lifestyle in Britain}
{the town-and-country planning in Britain}



\end{enumerate}


\newpage
\subsection{Text 3}


``There is one and only one social responsibility of business,''
Wrote Milton Friedman, a Nobel Prize-winning economist, ``That is, to
use its resources and engage in activities designed to increase its
profits.'' But even if you accept Friedman's premise and regard
corporate social responsibility (CSR) policies as a waste of
shareholders's money, things may not be absolutely clear-cut. New
research suggests that CSR may create monetary value for companies---at least when they are prosecuted for corruption.

The largest firms in America and Britain together spend more than
\$15 billion a year on CSR, according to an estimate by EPG, a
consulting firm. This could add value to their businesses in three
ways. First, consumers may take CSR spending as a ``signal'' that a
company's products are of high quality. Second, customers may be
willing to buy a company's products as an indirect may to donate to the
good causes it helps. And third, through a more diffuse ``halo
effect'' whereby its good deeds earn it greater consideration from
consumers and others.

Previous studies on CSR have had trouble differentiating these effects
because consumers can be affected by all three. A recent study attempts
to separate them by looking at bribery prosecutions under American's
Foreign Corrupt Practices Act (FCPA). It argues that since prosecutors
do not consume a company's products as part of their
investigations, they could be influenced only by the halo effect.

The study found that, among prosecuted firms, those with the most
comprehensive CSR programmes tended to get \uline{more lenient} penalties.
Their analysis ruled out the possibility that it was firm's political
influence, rather than their CSR stand, that accounted for the
leniency: Companies that contributed more to political campaigns did not
receive lower fines.

In all, the study concludes that whereas prosecutors should only
evaluate a case based on its merits, they do seem to be influenced by a
company's record in CSR. ``We estimate that either eliminating a
substantial labour-rights concern, such as child labour, or increasing
corporate giving by about 20\% result in fines that generally are 40\%
lower than the typical punishment for bribing foreign officials,'' says
one researcher.

Researchers admit that their study does not answer the question of
how much businesses ought to spend on CSR. Nor does it reveal how much
companies are banking on the halo effect, rather than the other
possible benefits, when they decide their do-gooding policies. But at
least they have demonstrated that when companies get into trouble with
the law, evidence of good character can win them a less costly
punishment.


\begin{enumerate}[resume]
	%\renewcommand{\labelenumi}{\arabic{enumi}.}
	% A(\Alph) a(\alph) I(\Roman) i(\roman) 1(\arabic)
	%设定全局标号series=example	%引用全局变量resume=example
	%[topsep=-0.3em,parsep=-0.3em,itemsep=-0.3em,partopsep=-0.3em]
	%可使用leftmargin调整列表环境左边的空白长度 [leftmargin=0em]
	\item
The author views Milton Friedman's statement about CSR
with \lineread.


\fourchoices
{tolerance}
{skepticism}
{uncertainty}
{approval}



\item
According to Paragraph 2, CSR helps a company by \lineread.


\fourchoices
{winning trust from consumers}
{guarding it against malpractices}
{protecting it from being defamed}
{raising the quality of its products}


\item
The expression ``more lenient'' (line 2, Para. 4) is
closest in meaning to \lineread.


\fourchoices
{more effective}
{less controversial}
{less severe}
{more lasting}


\item
 When prosecutors evaluate a case, a company's CSR
record \lineread.


\fourchoices
{has an impact on their decision}
{comes across as reliable evidence}
{increases the chance of being penalized}
{constitutes part of the investigation}


\item
Which of the following is true of CSR according to the
last paragraph?


\fourchoices
{Its negative effects on businesses are often overlooked}
{The necessary amount of companies spending on it is unknown}
{Companies' financial capacity for it has been overestimated}
{It has brought much benefit to the banking industry}


\end{enumerate}



\newpage
\subsection{Text 4}


There will eventually come a day when \emph{The New York Times} ceases to
publish stories on newsprint. Exactly when that day will be is a matter
of debate. ''Sometime in the future,'' the paper's publisher said back
in 2010.

Nostalgia for ink on paper and the rustle of pages aside, there's
plenty of incentive to ditch print. The infrastructure required to make
a physical newspaper---printing presses, delivery trucks---isn't just
expensive; it's excessive at a time when online-only competitors
don't have the same set of financial constraints. Readers are migrating
away from print anyway. And though print ad sales still dwarf their
online and mobile counterparts, revenue from print is still declining.

Overhead may be high and circulation lower, but rushing to
eliminate its print edition would be a mistake, says BuzzFeed CEO Jonah
Peretti.

Peretti says the \emph{Times} shouldn't waste time getting out of the
print business, but only if they go about doing it the right way.
``Figuring out a way to accelerate that transition would make sense for
them,'' he said, ``but if you discontinue it, you're going to have
your most loyal customers really upset with you.''

Sometimes that's worth making a change anyway. Peretti gives the
example of Netflix discontinuing its DVD--mailing service to focus on
streaming. ``It was seen as blunder,'' he said. The move turned out
to be foresighted. And if Peretti were in charge at the \emph{Times}? ``I
wouldn't pick a year to end print,'' he said. ``I would raise prices
and make it into more of a legacy product.''

The most loyal customers would still get the product they favor, the
idea goes, and they'd feel like they were helping sustain the quality
of something they believe in. ``So if you're overpaying for print, you
could feel like you were helping,'' Peretti said. ``Then increase it
at a higher rate each year and essentially try to generate additional
revenue.'' In other words, if you're going to make a print product,
make it for the people who are already obsessed with it. Which may be
what the Times is doing already. Getting the print edition seven days a
week costs nearly \$500 a year---more than twice as much as a digital--only subscription.

``It's a really hard thing to do and it's a tremendous luxury that
BuzzFeed doesn't have a legacy business,'' Peretti remarked. ``But
we're going to have questions like that where we have things we're doing
that don't make sense when the market changes and the world changes. In
those situations, it's better to be more aggressive than less
aggressive.''


\begin{enumerate}[resume]
	%\renewcommand{\labelenumi}{\arabic{enumi}.}
	% A(\Alph) a(\alph) I(\Roman) i(\roman) 1(\arabic)
	%设定全局标号series=example	%引用全局变量resume=example
	%[topsep=-0.3em,parsep=-0.3em,itemsep=-0.3em,partopsep=-0.3em]
	%可使用leftmargin调整列表环境左边的空白长度 [leftmargin=0em]
	\item
\emph{The New York Times} is considering ending its print
edition partly due to \lineread.


\fourchoices
{the high cost of operation}
{the pressure from its investors}
{the complaints from its readers}
{the increasing online ad sales}


\item
Peretti suggests that , in face of the present
situation, the \emph{Times} should \lineread.


\fourchoices
{seek new sources of readership}
{end the print edition for good}
{aim for efficient management}
{make strategic adjustments}

\item
 It can be inferred from Paragraphs 5 and 6 that a ``legacy
product'' \lineread.


\fourchoices
{helps restore the glory of former times}
{is meant for the most loyal customers}
{will have the cost of printing reduced}
{expands the popularity of the paper}



\item
Peretti believes that, in a changing world, \lineread.


\fourchoices
{legacy businesses are becoming outdated}
{cautiousness facilitates problem-solving}
{aggressiveness better meets challenges}
{traditional luxuries can stay unaffected}



\item
which of the following would be the best title of the
text?


\fourchoices
{Shift to Online Newspapers All at Once}
{Cherish the Newspapers Still in Your Hand}
{Make Your Print Newspapers a Luxury Good}
{Keep Your Newspapers Forever in Fashion}




\end{enumerate}

\newpage
\noindent
\textbf{Part B}\\
\textbf{ Directions:}\\
 Read the following text and answer the questions by choosing
	the most suitable subheading from the list A-G for each of the numbered
	paragraphs (41-45). There are two extra subheadings. Mark your answers
	on the \textbf{ANSER SHEET}. (10 point)

\begin{listmatch}
	%\renewcommand{\labelenumi}{\arabic{enumi}.}
	% A(\Alph) a(\alph) I(\Roman) i(\roman) 1(\arabic)
	%设定全局标号series=example	%引用全局变量resume=example
	%[topsep=-0.3em,parsep=-0.3em,itemsep=-0.3em,partopsep=-0.3em]
	%可使用leftmargin调整列表环境左边的空白长度 [leftmargin=0em]
	\item
Create a new image of yourself


\item 
 Decide if the time is right


\item 
 Have confidence in yourself


\item 
 Understand the context


\item 
 Work with professionals


\item 
 Make it efficient


\item 
 Know your goals
\end{listmatch}

No matter how formal or informal the work environment, the way you
present yourself has an impact. This is especially true in the first
impressions. According to research from Princeton University , people
assess your competence, trustworthiness, and likeability in just a
tenth of a second, solely based on the way you look.

The difference between today's workplace and the ``dress for
success'' era is that the range of options is so much broader. Norms
have evolved and fragmented. In some settings, red sneakers or dress
T-shirts can convey status; in others not so much. Plus, whatever
image we present is magnified by social-media services like LinkedIn.
Chances are, your headshots are seen much more often now than a decade
or two ago. Millennials, it seems, face the paradox of being the
least formal generation yet the most conscious of style and personal
branding. It can be confusing.

So how do we navigate this? How do we know when to invest in an
upgrade? And what's the best way to pull off one than enhances our
goals? Here are some tips:

\linefill.

As an executive coach, I've seen image upgrades be particularly
helpful during transitions-when looking for a new job, stepping into a
new or more public role, or changing work environments. If you're in a
period of change or just feeling stuck and in a rut, now may be a good
time. If you're not sure, ask for honest feedback from trusted
friends, colleagues and professionals. Look for cues about how others
perceive you. Maybe there's no need for an upgrade and that's OK.

\linefill.

Get clear on what impact you're hoping to have. Are you looking to
refresh your image or pivot it? For one person, the goal may be to be
taken more seriously and enhance their professional image. For
another, it may be to be perceived as more approachable, or more
modern and stylish. For someone moving from finance to advertising,
maybe they want to look more ``SoHo.'' (It's OK to use
characterizations like that.)

\linefill.

Look at your work environment like an anthropologist. What are the
norms of your environment? What conveys status? Who are your most
important audiences? How do the people you respect and look up to
present themselves? The better you understand the cultural context, the
more control you can have over your impact.

\linefill.

Enlist the support of professionals and share with them your goals
and context. Hire a personal stylist, or use the free styling service
of a store like J. Crew. Try a hair stylist instead of a barber. Work
with a professional photographer instead of your spouse or friend. It's
not as expensive as you might think.

\linefill.

The point of a style upgrade isn't to become more vain or to spend
more time fussing over what to wear. Instead, use it as an opportunity
to reduce decision fatigue. Pick a standard work uniform or a few go-to
options. Buy all your clothes at once with a stylist instead of
shopping alone, one article of clothing at a time.

\newpage
\noindent
\textbf{Part C}\\
\textbf{ Directions:}\\
Read the following text carefully and then translate the
	underlined segments into Chinese. Your translation should be written
	neatly on the \textbf{ANSWER SHEET}. (10 points)

\TiGanSpace


Mental health is our birthright. \transnum \uline{We don't have to learn
	how to be mentally healthy; it is built into us in the same way
	that our bodies know how to heal a cut or mend a broken bone.} Mental
health can't be learned, only reawakened. It is like the immune system
of the body, which under stress or through lack of nutrition or
exercise can be weakened, but which never leaves us. When we don't
understand the value of mental health and we don't know how to gain
access to it, mental health will remain hidden from us. 
\transnum \uline{Our mental health doesn't really go anywhere; like the sun
	behind a cloud, it can be temporarily hidden from view, but it is
	fully capable of being restored in an instant.}

Mental health is the seed that contains self-esteem---confidence in
ourselves and an ability to trust in our common sense. It allows us to
have perspective on our lives-the ability to not take ourselves too
seriously, to laugh at ourselves, to see the bigger picture, and to
see that things will work out. It's a form of innate or unlearned
optimism. \transnum \uline{Mental health allows us to view others with
	sympathy if they are having troubles, with kindness if they are in
	pain, and with unconditional love no matter who they are.} Mental
health is the source of creativity for solving problems, resolving
conflict, making our surroundings more beautiful, managing our home
life, or coming up with a creative business idea or invention to make
our lives easier. It gives us patience for ourselves. and toward
others as well as patience while driving, catching a fish, working on
our car, or raising a child. It allows us to see the beauty that
surrounds us each moment in nature, in culture, in the flow of our
daily lives.

\transnum \uline{Although mental health is the cure-all for living our
	lives, it is perfectly ordinary as you will see that it has been there
	to direct you through all your difficult decisions.} It has been
available even in the most mundane of life situations to show you right
from wrong, good from bad, friend from foe. Mental health has
commonly been called conscience, instinct, wisdom, common sense, or
the inner voice, We think of it simply as a health and helpful flow of
intelligent thought. \transnum \uline{As you will come to see, knowing
	that mental health is always available and knowing to trust it allow us
	to slow down to the moment and live life happily.}


\section{Writing}


\noindent
\textbf{Part A}\\
\textbf{ 51. Directions:}

Suppose you are a librarian in your university. Write a notice of
about 100 words. providing the newly-enrolled international students
with relevant information about the library.

You should write neatly on the ANSWER SHEET.

\textbf{Do not} sign your own name at the end of the notice. Use Li
Ming instead.

\textbf{Do not} write the address. (10 points)

\vspace{2em}

\noindent
\textbf{Part B}\\
\textbf{ 52. Directions:}

Write an essay of 160-200 words based on the following pictures. In
your essay, you should
\begin{listwrite}
	%\renewcommand{\labelenumi}{\arabic{enumi}.}
	% A(\Alph) a(\alph) I(\Roman) i(\roman) 1(\arabic)
	%设定全局标号series=example	%引用全局变量resume=example
	%[topsep=-0.3em,parsep=-0.3em,itemsep=-0.3em,partopsep=-0.3em]
	%可使用leftmargin调整列表环境左边的空白长度 [leftmargin=0em]
	\item
 describe the pictures briefly

\item 
 interpret the meaning , and

\item 
 give your comments
\end{listwrite}

You should write neatly on the ANSWER SHEET. (20 points)


\begin{figure}[h!]
	\centering
	\includegraphics[width=0.87\linewidth]{picture/2016.png}
	\caption*{与其只提要求,不如做个榜样}
\end{figure}



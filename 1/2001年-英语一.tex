\bta{2001}

\section{Use of English}

\noindent
\textbf{Directions:}\\
For each numbered blank in the following passage, there are four
	choices marked A, B,
	C. and
	D.  Choose the best
	one and mark your answer on ANSWER SHEET 1 by blackening the
	corresponding letter in the brackets with a pencil. (10 points)


\TiGanSpace

The government is to ban payments to witnesses by newspapers seeking to
buy up people involved in prominent cases \cloze the trial of
Rosemary West.

In a significant \cloze of legal controls over the press, Lord
Irvine, the Lord Chancellor, will introduce a \cloze bill that
will propose making payments to witnesses \cloze and will
strictly control the amount of \cloze that can be given to a case
\cloze a trial begins.

In a letter to Gerald Kaufman, chairman of the House of Commons media
select committee, Lord Irvine said he \cloze with a committee
report this year which said that self regulation did not \cloze
sufficient control.

\cloze of the letter came two days after Lord Irvine caused a
\cloze of media protest when he said the \cloze of
privacy controls contained in European legislation would be left to
judges \cloze to Parliament.

The Lord Chancellor said introduction of the Human Rights Bill, which
\cloze the European Convention on Human Rights legally
\cloze in Britain, laid down that everybody was \cloze
to privacy and that public figures could go to court to protect
themselves and their families.

``Press freedoms will be in safe hands \cloze our British
judges,'' he said.

Witness payments became an \cloze after West was sentenced to 10
life sentences in 1995. Up to 19 witnesses were \cloze to have
received payments for telling their stories to newspapers. Concerns were
raised \cloze witnesses might be encouraged exaggerate their
stories in court to \cloze guilty verdicts.


\newpage
\begin{enumerate}
	%\renewcommand{\labelenumi}{\arabic{enumi}.}
	% A(\Alph) a(\alph) I(\Roman) i(\roman) 1(\arabic)
	%设定全局标号series=example	%引用全局变量resume=example
	%[topsep=-0.3em,parsep=-0.3em,itemsep=-0.3em,partopsep=-0.3em]
	%可使用leftmargin调整列表环境左边的空白长度 [leftmargin=0em]
	\item

\fourchoices
{as to}
{for instance}
{in particular}
{such as}




\item


\fourchoices
{tightening}
{intensifying}
{focusing}
{fastening}




\item


\fourchoices
{sketch}
{rough}
{preliminary}
{draft}




\item


\fourchoices
{illogical}
{illegal}
{improbable}
{improper}




\item


\fourchoices
{publicity}
{penalty}
{popularity}
{peculiarity}




\item


\fourchoices
{since}
{if}
{before}
{as}




\item


\fourchoices
{sided}
{shared}
{complied}
{agreed}




\item


\fourchoices
{present}
{offer}
{manifest}
{indicate}




\item


\fourchoices
{Release}
{Publication}
{Printing}
{Exposure}




\item


\fourchoices
{storm}
{rage}
{flare}
{flash}




\item

\fourchoices
{translation}
{interpretation}
{exhibition}
{demonstration}



\item


\fourchoices
{better than}
{other than}
{rather than}
{sooner than}




\item


\fourchoices
{changes}
{makes}
{sets}
{turns}




\item


\fourchoices
{binding}
{convincing}
{restraining}
{sustaining}




\item


\fourchoices
{authorized}
{credited}
{entitled}
{qualified}




\item


\fourchoices
{with}
{to}
{from}
{by}




\item


\fourchoices
{impact}
{incident}
{inference}
{issue}




\item


\fourchoices
{stated}
{remarked}
{said}
{told}




\item

\fourchoices
{what}
{when}
{which}
{that}




\item


\fourchoices
{assure}
{confide}
{ensure}
{guarantee}

\end{enumerate}

\vfil

\section{ Reading Comprehension}

\noindent
\textbf{Directions:}\\
Each of the passages below is followed by some questions. For
each questions there are four answers marked A, B,
C.
and
D.  Read the passages carefully and choose the best answer to
each of the questions. Then mark your answer on ANSWER SHEET 1 by
blackening the corresponding letter in the brackets with a pencil. (40
points)


\newpage

\subsection{Passage 1}

Specialisation can be seen as a response to the problem of an increasing
accumulation of scientific knowledge. By splitting up the subject matter
into smaller units, one man could continue to handle the information and
use it as the basis for further research. But specialisation was only
one of a series of related developments in science affecting the process
of communication. Another was the growing professionalisation of
scientific activity.

No clear-cut distinction can be drawn between professionals and amateurs
in science: exceptions can be found to any rule. Nevertheless, the word ``amateur'' does carry a connotation that the person concerned is not
fully integrated into the scientific community and, in particular, may
not fully share its values. The growth of specialisation in the
nineteenth century, with its consequent requirement of a longer, more
complex training, implied greater problems for amateur participation in
science. The trend was naturally most obvious in those areas of science
based especially on a mathematical or laboratory training, and can be
illustrated in terms of the development of geology in the United
Kingdom.

A comparison of British geological publications over the last century
and a half reveals not simply an increasing emphasis on the primacy of
research, but also a changing definition of what constitutes an
acceptable research paper. Thus, in the nineteenth century, local
geological studies represented worthwhile research in their own right;
but, in the twentieth century, local studies have increasingly become
acceptable to professionals only if they incorporate, and reflect on,
the wider geological picture. Amateurs, on the other hand, have
continued to pursue local studies in the old way. The overall result has
been to make entrance to professional geological journals harder for
amateurs, a result that has been reinforced by the widespread
introduction of refereeing, first by national journals in the nineteenth
century and then by several local geological journals in the twentieth
century. As a logical consequence of this development, separate journals
have now appeared aimed mainly towards either professional or amateur
readership. A rather similar process of differentiation has led to
professional geologists coming together nationally within one or two
specific societies, whereas the amateurs have tended either to remain in
local societies or to come together nationally in a different way.

Although the process of professionalisation and specialisation was
already well under way in British geology during the nineteenth century,
its full consequences were thus delayed until the twentieth century. In
science generally, however, the nineteenth century must be reckoned as
the crucial period for this change in the structure of science.


\begin{enumerate}[resume]
	%\renewcommand{\labelenumi}{\arabic{enumi}.}
	% A(\Alph) a(\alph) I(\Roman) i(\roman) 1(\arabic)
	%设定全局标号series=example	%引用全局变量resume=example
	%[topsep=-0.3em,parsep=-0.3em,itemsep=-0.3em,partopsep=-0.3em]
	%可使用leftmargin调整列表环境左边的空白长度 [leftmargin=0em]
	\item
The growth of specialisation in the 19th century might be more
clearly seen in sciences such as \lineread.

\fourchoices
{sociology and chemistry}
{physics and psychology}
{sociology and psychology}
{physics and chemistry}



\item
We can infer from the passage that \lineread.


\fourchoices
{there is little distinction between specialisation and professionalisation}
{amateurs can compete with professionals in some areas of science}
{professionals tend to welcome amateurs into the scientific community}
{amateurs have national academic societies but no local ones}



\item
The author writes of the development of geology to demonstrate \lineread.


\fourchoices
{the process of specialisation and professionalisation}
{the hardship of amateurs in scientific study}
{the change of policies in scientific publications}
{the discrimination of professionals against amateurs}



\item
The direct reason for specialisation is \lineread.


\fourchoices
{the development in communication}
{the growth of professionalisation}
{the expansion of scientific knowledge}
{the splitting up of academic societies}

\end{enumerate}

\newpage

\subsection{Passage 2}

A great deal of attention is being paid today to the so-called digital
divide---the division of the world into the info (information) rich and
the info poor. And that divide does exist today. My wife and I lectured
about this looming danger twenty years ago. What was less visible then,
however, were the new, positive forces that work against the digital
divide. There are reasons to be optimistic.

There are technological reasons to hope the digital divide will narrow.
As the Internet becomes more and more commercialized, it is in the
interest of business to universalize access---after all, the more people
online, the more potential customers there are. More and more
governments, afraid their countries will be left behind, want to spread
Internet access. Within the next decade or two, one to two billion
people on the planet will be netted together. As a result, I now believe
the digital divide will narrow rather than widen in the years ahead. And
that is very good news because the Internet may well be the most
powerful tool for combating world poverty that we've ever had.

Of course, the use of the Internet isn't the only way to defeat poverty.
And the Internet is not the only tool we have. But it has enormous
potential.

To take advantage of this tool, some impoverished countries will have to
get over their outdated anti-colonial prejudices with respect to foreign
investment. Countries that still think foreign investment is an invasion
of their sovereignty might well study the history of infrastructure (the
basic structural foundations of a society) in the United States. When
the United States built its industrial infrastructure, it didn't have
the capital to do so. And that is why America's Second Wave
infrastructure---including roads, harbors, highways, ports and so on---were
built with foreign investment. The English, the Germans, the Dutch and
the French were investing in Britain's former colony. They financed
them. Immigrant Americans built them. Guess who owns them now? The
Americans. I believe the same thing would be true in places like Brazil
or anywhere else for that matter. The more foreign capital you have
helping you build your Third Wave infrastructure, which today is an
electronic infrastructure, the better off you're going to be. That
doesn't mean lying down and becoming fooled, or letting foreign
corporations run uncontrolled. But it does mean recognizing how
important they can be in building the energy and telecom infrastructures
needed to take full advantage of the Internet.


\begin{enumerate}[resume]
	%\renewcommand{\labelenumi}{\arabic{enumi}.}
	% A(\Alph) a(\alph) I(\Roman) i(\roman) 1(\arabic)
	%设定全局标号series=example	%引用全局变量resume=example
	%[topsep=-0.3em,parsep=-0.3em,itemsep=-0.3em,partopsep=-0.3em]
	%可使用leftmargin调整列表环境左边的空白长度 [leftmargin=0em]
	\item
Digital divide is something \lineread.


\fourchoices
{getting worse because of the Internet}
{the rich countries are responsible for}
{the world must guard against}
{considered positive today}


\item
 Governments attach importance to the Internet because it \lineread.


\fourchoices
{offers economic potentials}
{can bring foreign funds}
{can soon wipe out world poverty}
{connects people all over the world}


\item
The writer mentioned the case of the United States to justify the
policy of \lineread.


\fourchoices
{providing financial support overseas}
{preventing foreign capital's control}
{building industrial infrastructure}
{accepting foreign investment}



\item
It seems that now a country's economy depands much on \lineread.


\fourchoices
{how well-developed it is electronically}
{whether it is prejudiced against immigrants}
{whether it adopts America's industrial pattern}
{how much control it has over foreign corporations}
	
\end{enumerate}



\newpage
\subsection{Passage 3}

Why do so many Americans distrust what they read in their newspapers?
The American Society of Newspaper Editors is trying to answer this
painful question. The organization is deep into a long self-analysis
known as the journalism credibility project.

Sad to say, this project has turned out to be mostly low-level findings
about factual errors and spelling and grammar mistakes, combined with
lots of head-scratching puzzlement about what in the world those readers
really want.

But the sources of distrust go way deeper. Most journalists learn to see
the world through a set of standard templates (patterns) into which they
plug each day's events. In other words, there is a conventional story
line in the newsroom culture that provides a backbone and a ready-made
narrative structure for otherwise confusing news.

There exists a social and cultural disconnect between journalists and
their readers which helps explain why the ``standard templates''of the
newsroom seem alien to many readers. In a recent survey, questionnaires
were sent to reporters in five middle size cities around the country,
plus one large metropolitan area. Then residents in these communities
were phoned at random and asked the same questions.

Replies show that compared with other Americans, journalists are more
likely to live in upscale neighborhoods, have maids, own Mercedeses, and
trade stocks, and they're less likely to go to church, do volunteer
work, or put down roots in community.

Reporters tend to be part of a broadly defined social and cultural
elite, so their work tends to reflect the conventional values of this
elite. The astonishing distrust of the news media isn't rooted in
inaccuracy or poor reportorial skills but in the daily clash of world
views between reporters and their readers.

This is an explosive situation for any industry, particularly a
declining one. Here is a troubled business that keeps hiring employees
whose attitudes vastly annoy the customers. Then it sponsors lots of
symposiums and a credibility project dedicated to wondering why
customers are annoyed and fleeing in large numbers. But it never seems
to get around to noticing the cultural and class biases that so many
former buyers are complaining about. If it did, it would open up its
diversity program, now focused narrowly on race and gender, and look for
reporters who differ broadly by outlook, values, education, and class.


\begin{enumerate}[resume]
	%\renewcommand{\labelenumi}{\arabic{enumi}.}
	% A(\Alph) a(\alph) I(\Roman) i(\roman) 1(\arabic)
	%设定全局标号series=example	%引用全局变量resume=example
	%[topsep=-0.3em,parsep=-0.3em,itemsep=-0.3em,partopsep=-0.3em]
	%可使用leftmargin调整列表环境左边的空白长度 [leftmargin=0em]
	\item
What is the passage mainly about?


\fourchoices
{Needs of the readers all over the world.}
{Causes of the public disappointment about newspapers.}
{Origins of the declining newspaper industry.}
{Aims of a journalism credibility project.}



\item
The results of the journalism credibility project turned out to be \lineread.


\fourchoices
{quite trustworthy}
{somewhat contradictory}
{very illuminating}
{rather superficial}


\item
The basic problem of journalists as pointed out by the writer lies
in their \lineread.


\fourchoices
{working attitude}
{conventional lifestyle}
{world outlook}
{educational background}


\item
Despite its efforts, the newspaper industry still cannot satisfy the
readers owing to its \lineread.


\fourchoices
{failure to realize its real problem}
{tendency to hire annoying reporters}
{likeliness to do inaccurate reporting}
{prejudice in matters of race and gender}


\end{enumerate}


\newpage
\subsection{Passage 4}

The world is going through the biggest wave of mergers and acquisitions
ever witnessed. The process sweeps from hyperactive America to Europe
and reaches the emerging countries with unsurpassed might. Many in these
countries are looking at this process and worrying: " Won't the wave of
business concentration turn into an uncontrollable anti-competitive
force?"

There's no question that the big are getting bigger and more powerful.
Multinational corporations accounted for less than 20\% of international
trade in 1982. Today the figure is more than 25\% and growing rapidly.
International affiliates account for a fast-growing segment of
production in economies that open up and welcome foreign investment. In
Argentina, for instance, after the reforms of the early 1990 s,
multinationals went from 43\% to almost 70\% of the industrial
production of the 200 largest firms. This phenomenon has created serious
concerns over the role of smaller economic firms, of national
businessmen and over the ultimate stability of the world economy.

I believe that the most important forces behind the massive M\&A wave
are the same that underlie the globalization process: falling
transportation and communication costs, lower trade and investment
barriers and enlarged markets that require enlarged operations capable
of meeting customers' demands. All these are beneficial, not
detrimental, to consumers. As productivity grows, the world's wealth
increases.

Examples of benefits or costs of the current concentration wave are
scanty. Yet it is hard to imagine that the merger of a few oil firms
today could re-create the same threats to competition that were feared
nearly a century ago in the U.S., when the Standard Oil trust was broken
up. The mergers of telecom companies, such as WorldCom, hardly seem to
bring higher prices for consumers or a reduction in the pace of
technical progress. On the contrary, the price of communications is
coming down fast. In cars, too, concentration is increasing---witness
Daimler and Chrysler, Renault and Nissan-but it does not appear that
consumers are being hurt.

Yet the fact remains that the merger movement must be watched. A few
weeks ago, Alan Greenspan warned against the megamergers in the banking
industry. Who is going to supervise, regulate and operate as lender of
last resort with the gigantic banks that are being created? Won't
multinationals shift production from one place to another when a nation
gets too strict about infringements to fair competition? And should one
country take upon itself the role of ``defending competition'' on issues
that affect many other nations, as in the U.S. vs. Microsoft case?


\begin{enumerate}[resume]
	%\renewcommand{\labelenumi}{\arabic{enumi}.}
	% A(\Alph) a(\alph) I(\Roman) i(\roman) 1(\arabic)
	%设定全局标号series=example	%引用全局变量resume=example
	%[topsep=-0.3em,parsep=-0.3em,itemsep=-0.3em,partopsep=-0.3em]
	%可使用leftmargin调整列表环境左边的空白长度 [leftmargin=0em]
	\item
What is the typical trend of businesses today?


\fourchoices
{To take in more foreign funds.}
{To invest more abroad.}
{To combine and become bigger.}
{To trade with more countries.}


\item
According to the author, one of the driving forces behind M\&A wave
is \lineread


\fourchoices
{the greater customer demands.}
{a surplus supply for the market.}
{a growing productivity.}
{the increase of the world's wealth.}


\item
From paragraph 4 we can infer that \lineread.


\fourchoices
{the increasing concentration is certain to hurt consumers}
{WorldCom serves as a good example of both benefits and costs}
{the costs of the globalization process are enormous}
{the Standard Oil trust might have threatened competition}



\item
Toward the new business wave, the writer's attitude can he said to
be \lineread.


\fourchoices
{optimistic}
{objective}
{pessimistic}
{biased}

	
\end{enumerate}



\newpage
\subsection{Passage 5}

When I decided to quit my full time employment it never occurred to me
that I might become a part of a new international trend. A lateral move
that hurt my pride and blocked my professional progress prompted me to
abandon my relatively high profile career although, in the manner of a
disgraced government minister, I covered my exit by claiming ``I wanted
to spend more time with my family''.

Curiously, some two-and-a-half years and two novels later, my experiment
in what the Americans term ``downshifting''has turned my tired excuse
into an absolute reality. I have been transformed from a passionate
advocate of the philosophy of ``having it all'', preached by Linda
Kelsey for the past seven years in the pages of She magazine, into a
woman who is happy to settle for a bit of everything.

I have discovered, as perhaps Kelsey will after her much-publicized
resignation from the editorship of She after a build-up of stress, that
abandoning the doctrine of ``juggling your life'', and making the
alternative move into ``downshifting'' brings with it far greater
rewards than financial success and social status. Nothing could persuade
me to return to the kind of life Kelsey used to advocate and I once
enjoyed: 12-hour working days, pressured deadlines, the fearful strain
of office politics and the limitations of being a parent on ``quality
time''.

In America, the move away from juggling to a simpler, less materialistic
lifestyle is a well-established trend. Downshifting-also known in
America as ``voluntary simplicity''---has, ironically, even bred a new
area of what might be termed anticonsumerism. There are a number of
best-selling downshifting self-help books for people who want to simplify
their lives; there are newsletter's, such as The Tightwad Gazette, that
give hundreds of thousands of Americans useful tips on anything from
recycling their cling-film to making their own soap; there are even
support groups for those who want to achieve the mid- '90s equivalent of
dropping out.

While in America the trend started as a reaction to the economic
decline---after the mass redundancies caused by downsizing in the
late '80s---and is still linked to the politics of thrift, in Britain,
at least among the middle-class downshifters of my acquaintance, we have
different reasons for seeking to simplify our lives.

For the women of my generation who were urged to keep juggling through
the'80 s, downshifting in the mid-'90s is not so much a search for the
mythical good life---growing your own organic vegetables, and risking
turning into one---as a personal recognition of your limitations.

\begin{enumerate}[resume]
	%\renewcommand{\labelenumi}{\arabic{enumi}.}
	% A(\Alph) a(\alph) I(\Roman) i(\roman) 1(\arabic)
	%设定全局标号series=example	%引用全局变量resume=example
	%[topsep=-0.3em,parsep=-0.3em,itemsep=-0.3em,partopsep=-0.3em]
	%可使用leftmargin调整列表环境左边的空白长度 [leftmargin=0em]
	\item
Which of the following is true according to paragraph 1?


\fourchoices
{Full-time employment is a new international trend.}
{The writer was compelled by circumstances to leave her job.}
{``A lateral move'' means stepping out of full-time employment.}
{The writer was only too eager to spend more time with her family.}



\item
The writer's experiment shows that downshifting \lineread


\fourchoices
{enables her to realize her dream}
{helps her mold a new philosophy of life}
{prompts her to abandon her high social status}
{leads her to accept the doctrine of \emph{She} magazine}


\item
``Juggling one's life'' probably means living a life characterized
by \lineread.


\fourchoices
{non-materialistic lifestyle}
{a bit of everything}
{extreme stress}
{anti-consumerism}


\item
According to the passage, downshifting emerged in the U.S. as a
result of \lineread


\fourchoices
{the quick pace of modern life}
{man's adventurous spirit}
{man's search for mythical experiences}
{the economic situation}

	
\end{enumerate}


\newpage
\section{ English-Chinese Translation}

\noindent
\textbf{Directions:}\\
Read the following text carefully and then translate the
underlined segments into Chinese. Your translation should be written
clearly on ANSWER SHEET 2. (15 points)

\TiGanSpace

In less than 30 years' time the \emph{Star Trek} holodeck will be a reality.
Direct links between the brain's nervous system and a computer will also
create full sensory virtual environments, allowing virtual vacations
like those in the film \emph{Total Recall}.



\transnum \uline{There will be television chat shows hosted by robots, and
cars with pollution monitors that will disable them when they offend}.
\transnum \uline{Children will play with dolls equipped with personality
chips, computers with in-built personalities will be regarded as
workmates rather than tools, relaxation will be in front of smell-television, and digital age will have arrived.}

According to BT's futurologist, Ian Pearson, these are among the
developments scheduled for the first few decades of the new
millennium (a period of 1,000 years), when supercomputers will
dramatically accelerate progress in all areas of life.

\transnum \uline{Pearson has pieced together the work of hundreds of
researchers around the world to produce a unique millennium technology
calendar that gives the latest dates when we can expect hundreds of key
breakthroughs and discoveries to take place.} Some of the biggest
developments will be in medicine, including an extended life expectancy
and dozens of artificial organs coming into use between now and 2040.

Pearson also predicts a breakthrough in computer-human links. ``By
linking directly to our nervous system, computers could pick up what we
feel and, hopefully, simulate feeling too so that we can start to
develop full sensory environments, rather like the holidays in \emph{Total Recall} or the \emph{Star Trek} holodeck,'' he says. \transnum \uline{But that,
Pearson points out, is only the start of man-machine integration: ``It
will be the beginning of the long process of integration that will
ultimately lead to a fully electronic human before the end of the next
century}.''

Through his research, Pearson is able to put dates to most of the
breakthroughs that can be predicted. However, there are still no
forecasts for when faster-than-light travel will be available, or when
human cloning will be perfected, or when time travel will be possible.
But he does expect social problems as a result of technological
advances. A boom in neighborhood surveillance cameras will, for example,
cause problems in 2010, while the arrival of synthetic lifelike robots
will mean people may not be able to distinguish between their human
friends and the droids.
\transnum \uline{And home appliances will also become so smart that
controlling and operating them will result in the breakout of a new
psychological disorder---kitchen rage.}




\section{Writing}


\noindent
\textbf{46. Directions:}

Among all the worthy feelings of mankind, love is probably the noblest,
but everyone has his/her own understanding of it.

There has been a discussion recently on the issue in a newspaper. Write
an essay to the newspaper to
\begin{listwrite}
	\item 
show your understanding of the symbolic meaning of the picture below,

\item 
give a specific example, and

\item 
give your suggestion as to the best way to show love.

\end{listwrite}

You should write about 200 words neatly on ANSWER SHEET 2. (20 points)

\begin{figure}[h!]
	\centering
	\includegraphics[width=0.5\linewidth]{picture/2001.png}
\end{figure}



